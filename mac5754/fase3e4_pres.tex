\documentclass{beamer}
\usepackage[brazilian]{babel}
\usepackage[utf8]{inputenc}
\usepackage[T1]{fontenc}
\usepackage{amsfonts}

\AtBeginSection{\frame{\sectionpage}}
\newtranslation[to=brazilian]{Section}{Se\c c\~ao}

\newtheorem{mydef}{Defini\c c\~ao}


\title{Projeto da Linguagem de Programa\c c\~ao II}
\author{Lucas Virgili}
\date{}

\begin{document}

\begin{frame}[fragile]
  \titlepage
\end{frame}

\begin{frame}[fragile]
  \frametitle{Relembrando}
  \begin{itemize}
  \item O objetivo da linguagem \'e facilitar o c\'alculo de opera\c
    c\~oes comuns quando trabalhamos com sequ\^encias e s\'eries, tais
    como calcular termos e limites.
  \end{itemize}
\end{frame}

\section{Fase 3}

\begin{frame}[fragile]
  \frametitle{Palavras reservadas}
{\small
\begin{table}[h]
\begin{center}
\begin{tabular}{|c|c|}
\hline
\textbf{Palavra}             & \textbf{Significado}                                                         \\ \hline
program             & Usado para iniciar o programa \\ \hline
begin               & In\'icio de um bloco \\ \hline
end                 & Fim de um bloco \\ \hline
is                  & Espeficicador de in\'icio de bloco \\ \hline
are                 & Especificador de in\'icio de bloco de declara\c c\~oes \\ \hline
declarations        & Usado para iniciar as declara\c c\~oes do programa \\ \hline
int                 & Usado para declarar vari\'aveis do tipo inteiro                     \\ \hline
real                & Usado para declarar vari\'aveis do tipo real                        \\ \hline
%sequence           & Calcula o en\'esimo termo de uma sequ\^encia           \\ \hline
take                & Retorna elementos de uma sequ\^encia                \\ \hline
series              & Calcula a soma parcial de uma sequ\^encia           \\ \hline
sequence\_converges & Checa se uma sequ\^encia converge                   \\ \hline
series\_converges   & Checa se uma s\'erie converge                   \\ \hline
$\sin$              & Fun\c c\~ao seno                                                    \\ \hline
$\cos$              & Fun\c c\~ao cosseno                                                 \\ \hline
$\tan$              & Fun\c c\~ao tangente                                                \\ \hline
$\log$              & Fun\c c\~ao logaritmo com base $e$                                  \\ \hline
$\exp$              & Fun\c c\~ao exponencial                                             \\ \hline
sqrt                & Fun\c c\~ao ra\'iz quadrada                                         \\ \hline
\end{tabular}
\end{center}
\end{table}
}
\end{frame}
\begin{frame}[fragile]
  \frametitle{Regras l\'exicas}

\begin{itemize}
\item Caracteres:
\begin{verbatim}
<letra> = "a" | "A" | "b" | "B" | "c" | "C" | ...
| "W" | "x" | "X" | "y" | "Y" | "z" | "Z";
<char> = <letra> | "_";
\end{verbatim}
\item N\'umeros:
\begin{verbatim}
<dig> = 0 | 1 | 2 | 3 | 4 | 5 | 6 | 7 | 8 | 9;
<dig_sem_zero> = 1 | 2 | 3 | 4 | 5 | 6 | 7 | 8 | 9;
<int> = [ "-" ], <dig_sem_zero>, { <dig> };
<real> = [ "-" ], <int>, [ "."{ dig } ];
<numero> = <int> | <real>;
\end{verbatim}

\item Vari\'aveis:
\begin{verbatim}
<id> = <char>{ <char> | <dig> };
\end{verbatim}

\item Literais:
\begin{verbatim}
<lit> = <int> | <real>;
\end{verbatim}
\end{itemize}
\end{frame}

\begin{frame}[fragile]
\frametitle{Regras sint\'aticas I}
\begin{verbatim}
(* Definicao da linguagem *)

(* Definicao de um programa *)
<prog> = `program' <id> `is' `begin' `declararions' `are'
`begin'<declaracoes> `end';
<seq_expressoes>
`end'

<seq_expressoes> = <expressao> | <seq_expressoes>;

(* Comandos *)
(* Declaracoes. Tipos possiveis sao int e real *)
<tipo> = `int' | `real';
<declaracoes> = {<tipo> <id>; | <sequence>};
<sequence> = (`sequence' <id> = <seq_expressoes>)
\end{verbatim}
\end{frame}

\begin{frame}[fragile]
\frametitle{Regras sint\'aticas II}
\begin{verbatim}
(* Funcoes builtin *)
<func> = ( sin | cos | tan | log | exp | sqrt ) `(' <expressao> `)';

(* take *)
<take> = take `(' <id> , <int> `)';

(* series *)
<series> = series `(' <id> , <int> `)';

(* Operadores de Convergencia *)
<seq_conv> = sequence_converges `(' <id>, <real> `)';
<series_conv> = series_converges `(' <id>, <real> `)';
\end{verbatim}
\end{frame}

\begin{frame}[fragile]
\frametitle{Regras sint\'aticas III}
\begin{verbatim}
(* Atribuicoes *)
<atr> = (<id> = <expressao>);

<cmd> = <atr> | <take> | <series>
        | <seq_conv> | <series_conv>;

(* Definicao de uma expressao *)
<expressao> = ( <expressao> + <termo> ) |
              ( <expressao> - <termo> ) |
              ( <termo> );

<termo> = ( <termo> * <fator> ) |
          ( <termo> / <fator> ) |
          ( <fator> );

<fator> = <id> | <cmd> | <func> | <numero> | <expressao>;
\end{verbatim}
\end{frame}

\begin{frame}[fragile]
\frametitle{Exemplo de programa}
\begin{verbatim}
program X is
begin
    declarations are
    begin
            real pi;
            sequence s is
            begin
                pi / n ^ 2;
            end
    end
    pi = 3.1415;
    s(3);
    series(s, 3);
    sequence_converges(s, 0.000001);
    series_converges(s, 0.00001)
end
\end{verbatim}
\end{frame}

\section{Fase 4}
\begin{frame}[fragile]
\frametitle{Dom\'inios sem\^anticos}
\begin{description}
\item[Estados]\hfill
  \begin{itemize}
  \item $\Sigma = \{\sigma, \sigma '\}$
  \end{itemize}
\item[N\'umeros]\hfill
  \begin{itemize}
  \item Inteiros = $\mathbb{Z}$
  \item Reais = $\mathbb{R}$
  \end{itemize}
\item[Valores]\hfill
  \begin{itemize}
  \item $Valores = Inteiros \cup Reais$
  \end{itemize}
\item[Fun\c c\~oes \emph{built-in}]\hfill
  \begin{itemize}
  \item $F = \{\sin, \cos, \tan, \log, \exp,$ $sqrt$, \\$series, take, sequence\_converges, series\_converges\}$
  \end{itemize}
\item[Identificadores]\hfill
  \begin{itemize}
  \item $id = \{X, Y, Z, \ldots\}$
  \end{itemize}
\item[Express\~oes]\hfill
  \begin{itemize}
  \item $exp = \{a_0, a_1, a_2, \ldots\}$
  \end{itemize}
\end{description}
\end{frame}

\begin{frame}[fragile]
\frametitle{Fun\c c\~oes sem\^anticas}
\begin{description}
\item[Express\~oes]\hfill
  \begin{itemize}
  \item $E: exp \to (\Sigma \to Valores)$
  \end{itemize}
\item[Comandos]\hfill
  \begin{itemize}
  \item $C: cmd \to (\Sigma \to \Sigma)$
  \end{itemize}
\item[Sequ\^encias (Fun\c c\~oes)]\hfill
  \begin{itemize}
  \item $S : (\{0, 1, \ldots\} \to \Sigma
    \to \Sigma)$
  \end{itemize}
\end{description}
\end{frame}

\begin{frame}[fragile]
\frametitle{Equa\c c\~oes sem\^anticas: Express\~oes Aritm\'eticas}
  \begin{itemize}
  \item Seja $n$ um n\'umero:\\
    $E[[n]] = \{(\sigma, n)\ |\ \sigma \in \Sigma\}$
  \item Seja $x$ um identificador:\\
    $E[[x]] = \{(\sigma, \sigma (x))\ |\ \sigma \in \Sigma\}$
  \item Seja $f$ uma fun\c c\~ao e $e$ uma express\~ao:\\
    $E[[f]] = \{(\sigma, \sigma (f(e)))\ |\ \sigma \in \Sigma\}$
  \item Adi\c c\~ao:\\
    $E[[a_0 + a_1]] = \{(\sigma, (n_0 + n_1))\ |\ (\sigma, n_0) \in
    E[[n_0]]\ \wedge\ (\sigma, n_1) \in E[[n_1]]\}$
  \item Subtra\c c\~ao:\\
    $E[[a_0 - a_1]] = \{(\sigma, (n_0- n_1))\ |\ (\sigma, n_0) \in
    E[[n_0]]\ \wedge\ (\sigma, n_1) \in E[[n_1]]\}$
  \item Multiplica\c c\~ao:\\
    $E[[a_0 * a_1]] = \{(\sigma, (n_0 * n_1))\ |\ (\sigma, n_0) \in
    E[[n_0]]\ \wedge\ (\sigma, n_1) \in E[[n_1]]\}$
  \item Divis\~ao:\\
    $E[[a_0 / a_1]] = \{(\sigma, (n_0 / n_1))\ |\ (\sigma, n_0) \in
    E[[n_0]]\ \wedge\ (\sigma, n_1) \in E[[n_1]]\}$
  \end{itemize}
\end{frame}

\begin{frame}[fragile]
\frametitle{Equa\c c\~oes sem\^anticas: Comandos}
  \begin{itemize}
  \item Atribui\c c\~ao:
    \begin{itemize}
    \item Atribui\c c\~ao de n\'umeros:\\
      $C[[X = a]] = \{(\sigma, \sigma[n / X])\ |\ \sigma \in \Sigma\
      \wedge\ n = E[[a]]\sigma\}$

    \item ``Atribui\c c\~ao'' de sequ\^encias:\\
      $C[[S \mbox{ = B}]] = \{(\sigma, \sigma[n / S]\ |\ \sigma \in \Sigma\
      \wedge\ n = E[[B]]\sigma)$
    \end{itemize}
  \item Composi\c c\~ao:\\
    $C[[c_0;\ c_1]] = C[[c_1]]\ \circ\ C[[c_0]]$
  \end{itemize}
\end{frame}
\end{document}