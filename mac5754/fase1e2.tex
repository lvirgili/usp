\documentclass[a4 paper, 12pt]{article}
\usepackage[brazilian]{babel}
\usepackage[utf8]{inputenc}
\usepackage[T1]{fontenc}
\usepackage{minted}
\usepackage{anysize}
\marginsize{3cm}{3cm}{2cm}{2cm}
\usemintedstyle{tango}
\newminted{cpp}{linenos, mathescape, frame=topline, numberblanklines=false}
\newmint{cpp}{frame=single}
\usepackage{algorithmic}
\usepackage[brazilian]{algorithm}

\renewcommand{\algorithmicrequire}{\textbf{Entrada:}}
\renewcommand{\algorithmicensure}{\textbf{Saída:}}
\renewcommand{\algorithmicend}{\textbf{fim}}
\renewcommand{\algorithmicif}{\textbf{se}}
\renewcommand{\algorithmicthen}{\textbf{então}}
\renewcommand{\algorithmicelse}{\textbf{else}}
\renewcommand{\algorithmicelsif}{\algorithmicelse\ \algorithmicif}
\renewcommand{\algorithmicendif}{\algorithmicend\ \algorithmicif}
\renewcommand{\algorithmicfor}{\textbf{para}}
\renewcommand{\algorithmicforall}{\textbf{para todos}}
\renewcommand{\algorithmicdo}{\textbf{faça}}
\renewcommand{\algorithmicendfor}{\algorithmicend}
\renewcommand{\algorithmicwhile}{\textbf{enquanto}}
\renewcommand{\algorithmicendwhile}{\algorithmicend}
\renewcommand{\algorithmicloop}{\textbf{loop}}
\renewcommand{\algorithmicendloop}{\algorithmicend\ \algorithmicloop}
\renewcommand{\algorithmicrepeat}{\textbf{repita}}
\renewcommand{\algorithmicuntil}{\textbf{enquanto}}
\renewcommand{\algorithmicprint}{\textbf{imprima}}
\renewcommand{\algorithmicreturn}{\textbf{retorne}}
\renewcommand{\algorithmictrue}{\textbf{verdadeiro}}
\renewcommand{\algorithmicfalse}{\textbf{falso}}


\renewcommand{\theFancyVerbLine}{
  \sffamily\textcolor[rgb]{0.5,0.5,0.5}{\scriptsize\arabic{FancyVerbLine}}}

\title{Projeto da Linguagem de Programa\c c\~ao}
\author{Lucas Virgili}
\date{}

\begin{document}
\maketitle
\tableofcontents
\section{Fase 1}
\subsection{Dom\'inio}
\subsubsection{Introdu\c c\~ao}
H\'a aproximadamente 2400 anos, Zeno de Elea abalou as funda\c
c\~oes da matem\'atica da \'epoca atrav\'es da proposi\c c\~ao de
diversos paradoxos. Um deles \'e muito famoso:

\begin{quotation}
  Um corredor nunca pode terminar uma corrida, j\'a que para isso,
  ele primeiro tem que andar metade do percurso, em seguida um quarto,
  depois um oitavo, e assim por diante, \emph{ad infinitum}.
\end{quotation}

Ap\'os 2000 anos, matem\'aticos dos s\'eculos XII e XIII deram
in\'icio \`a teoria de s\'eries infinitas. Nessa teoria, a no\c c\~ao
usual de soma, v\'alida para conjuntos finitos, \'e expandida para
cole\c c\~oes infinitas.

Dessa forma, o ``paradoxo'' de Zeno foi selecionado, j\'a que, naquele
contexto, a soma que representa as ``etapas'' que o corredor deve
percorrer \'e conhecida
\begin{equation}
  \frac{1}{2} + \frac{1}{4} + \ldots + \frac{1}{2^n} + \ldots
\end{equation}

e seu resultado \'e $1$.

\subsubsection{Descri\c c\~ao do dom\'inio}
A teoria de s\'eries e sequ\^encias n\~ao tem seu uso limitado a
mostrar como gregos mortos estavam errados; ela \'e extremamente
usada. Por exemplo, em an\'alise, utiliza-se sequ\^encias de fun\c
c\~oes para demonstrar os teoremas de converg\^encia mon\'otona de
Lesbegue. Esse teorema \'e important\'issimo em probabilidade, por
exemplo.

Definimos sequ\^encias e s\'eries abaixo.

\paragraph{Sequ\^encia:} Uma fun\c c\~ao $f$ cujo dom\'inio \'e o
conjunto dos inteiros positivos $1, 2, 3, \ldots$ \'e uma sequ\^encia
infinita. O valor $f(n)$ \'e o en\'esimo \emph{termo} da sequ\^encia.

\paragraph{S\'erie:} Dada uma sequ\^encia, podemos gerar uma nova
sequ\^encia somando termos sucessivos. Logo, se temos uma sequ\^encia
\begin{equation}
  a_1, a_2, \ldots, a_n, \ldots
\end{equation}
Podemos gerar as seguintes ``somas parciais'':
\begin{equation}
  s_1 = a_1, s_2 = a_1 + a_2,  s_3 = a_1 + a_2 + a_3
\end{equation}
e assim continuarmos at\'e a en\'esima soma parcial:
\begin{equation}
  s_n = a_1 + a_2 + a_3 + a_4 + \ldots + a_n = \sum_{i = 1}^{n}a_i
\end{equation}
A sequ\^encia ${s_n}$ das somas parciais \'e chamada de \emph{s\'erie
  infinita} ou, simplesmente \emph{s\'erie}, e \'e denotada por
\begin{equation}
  a_1 + a_2 + a_3 + \ldots, \textrm{  ou  } \sum_{i = 1}^{\infty}a_i
\end{equation}

Informalmente, dizemos que uma sequ\^encia converge se existe uma
quantidade $L$ para a qual a sequ\^encia se aproxima o quanto
quisermos\footnote{Formalmente, se para qualquer $\epsilon > 0$,
  existe um um n\'umero positivo $N$ tal que $|f(n) - L| < \epsilon$
  para qualquer $n \geq N$.}. Uma s\'erie, ent\~ao, converge se sua
sequ\^encia ${s_n}$ converge.

\'E comum, enquanto estamos trabalhando com sequ\^encias ou s\'eries,
escrevermos programas em uma linguagem como C para avaliar se uma
sequ\^encia ou s\'erie converge.

\subsection{Proposta da linguagem de programa\c c\~ao}
Para este projeto, propomos desenvolver uma linguagem de programa\c
c\~ao que permita a declara\c c\~ao de sequ\^encias e s\'eries, bem
como analisar a converg\^encia das mesmas atrav\'es de m\'etodos
conhecidos.

Por exemplo, ser\'a poss\'ivel para o programador definir uma
sequ\^encia e operar sobre elas:
\begin{verbatim}
seq s
s(n) = 1 / n ^ 2
s(3) = 0.125
series(s, 3) = 0.875 ## calcula s_3
sequence_converges(s, 0.000001)
## -> (true, 0)
## Se a sequencia converge com uma precisao de 10 ^ (-5)


series_converges(s, 0.00001)
## -> (true, 1)
## Se a serie converge com uma "precisao" de 10 ^ (-5)
\end{verbatim}
\subsection{Elementos essencias \`a linguagem}
Os seguintes elementos s\~ao fundamentais para a linguagem:
\begin{enumerate}
\item \textbf{Declara\c c\~oes}\\Nesta linguagem, as ``vari\'aveis''
  ser\~ao as sequ\^encias. Como visto no exemplo acima, o programador
  poder\'a declarar sequ\^encias utilizando a palavra reservada
  \textbf{seq}.

\item \textbf{Operadores}\\Sejam ${a_n}$ e ${b_n}$ duas sequ\^encias
  convergentes. \'E f\'acil mostrar que a s\'erie
  \begin{equation}
    \sum_{n = 1}^{\infty}(\alpha a_n + \beta b_nE)
  \end{equation}
  tamb\'em converge e seu limite \'e dado por
  \begin{equation}
    \alpha\sum_{n=1}^{\infty}a_n + \beta\sum_{n=1}^{\infty}b_n
  \end{equation}
  Assim, podemos multiplicar s\'eries convergentes por constantes
  num\'ericas e tamb\'em podemos somar e subtrair s\'eries
  convergentes. Logo, a linguagem ir\'a fornecer os operadores soma
  ($+$) e subtra\c c\~ao ($-$) entre s\'eries e o operador produto
  ($*$) entre uma constante e uma s\'erie.

\item \textbf{Fun\c c\~oes}\\A linguagem oferecer\'a ao programador as
  seguintes fun\c c\~oes:
  \begin{center}
    \begin{table}[H]
    \begin{tabular}{|l | l |}
      \hline
      Fun\c c\~ao & O que ela calcula \\
      \hline
      series(sequencia, n) & calcula a en\'esima soma parcial de sequencia \\
      sequence\_converges(sequenca, precisao) & diz se sequ\^encia
      converge com precis\~ao precisao \\
      series\_converges(serie, precisao) & diz se a s\'erie converge
      com precis\~ao precisao \\
      \hline
    \end{tabular}
\end{table}
  \end{center}
\end{enumerate}



\section{Fase 2}
\subsection{Tipos de dados}
\subsubsection{Tipos primitivos}
Os seguintes tipos de dados primitivos ser\~ao inclusos na linguagem:

\begin{itemize}
\item \textbf{Inteiros}: Como vimos na defini\c c\~ao de sequ\^encias
  acima, estas s\~ao indexadas por inteiros positivos. Logo, \'e
  natural que a linguagem ofere\c ca inteiros para o programador.
\item \textbf{Reais}: O uso de n\'umeros n\~ao inteiros
  tamb\'em \'e fundamental. Assim, a linguagem utilizar\'a ponto
  flutuante para representar tais valores.
\item \textbf{Booleanos}: Como vimos no exemplo acima, pretende-se que
  as fun\c c\~oes de converg\^encia retornem um par (booleano, limite)
  e, caso o valor booleano seja verdadeiro, limite \'e o valor para
  qual a s\'erie ou sequ\^encia converge. Dessa forma, \'e
  necess\'ario que haja o tipo booleano para armazenar os valores
  verdadeiro ou falso.
\end{itemize}

\subsubsection{Tipos compostos}
N\~ao parece necess\'ario oferecer nenhum tipo composto na linguagem.
\subsection{Express\~oes}
\subsubsection{Literais}
Para denotarmos o valor dos tipos descritos acima, usaremos as nota\c
c\~oes usuais: $1$ denota o inteiro de valor um e $1.0$ denota o valor
real um.

\subsection{Fun\c c\~oes}
A linguagem oferecer\'a as fun\c c\~oes trigonom\'etricas usuais, como
$\sin, \cos, \tan$, bem como logaritmos, exponencia\c c\~ao e ra\'iz
quadrada.

N\~ao s\'o, os operadores aritm\'eticos usuais, $+, -, /$ e $*$
ser\~ao fornecidos para manipularmos inteiros e reais. Al\'em disso,
como dito acima, s\'eries convergentes continuam convergentes quando
multiplicadas por um n\'umero ou somadas a outra s\'erie
convergente. Logo, os operadores $+$ e $-$ ser\~ao tamb\'em definidos
sobre s\'eries e o operador $*$ ser\'a definido sobre um escalar e uma
s\'erie.

\subsection{Ordem de avalia\c c\~ao}
A ordem de avalia\c c\~ao ser\'a a usual da matem\'atica: com maior
preced\^encia, temos opera\c c\~oes delimitadas por par\^enteses,
seguido de apli\c c\~ao de fun\c c\~oes, depois a multiplica\c c\~ao e
divis\~ao com igual precend\^encia e, por fim, por soma e subtra\c
c\~ao.

Por exemplo a express\~ao
\begin{equation}
   1 + 5 * 2 \textrm{\textasciicircum} 2
\end{equation}
ser\'a avaliada como 21. Por sua vez,
\begin{equation}
  7 + 4 * 3
\end{equation}
\'e igual a 19. J\'a
\begin{equation}
  ( 7 + 4 ) * 3
\end{equation}
\'e avaliada como 33.

\subsection{Comandos}
\subsubsection{Declara\c c\~ao}
Haver\'a tr\^es comandos de declara\c c\~ao: um para sequ\^encias,
\textbf{seq}, um para vari\'aveis inteiras e um para vari\'aveis
reais:
\begin{verbatim}
seq s; # declara uma sequencia s
int i; # declara um inteiro a
real pi; # declara um valor real pi
\end{verbatim}
\subsubsection{Atribui\c c\~ao}
Atribui\c c\~ao ser\'a feita com o operador \textbf{=}. Contudo,
quando declararmos uma sequ\^encia, precisamos identificar qual \'e a
``inc\'ognita'':
\begin{verbatim}
real pi;
pi = 3.14;

seq s;
s(n) = pi * (1 / n)
\end{verbatim}

\subsection{Estruturas de controle de fluxo}
\subsubsection{Condicionais}
A princ\'ipio, n\~ao vemos necessidade em incluir condicionais
``usuais'' como if... then ... else na linguagem.

\subsubsection{Repeti\c c\~ao}
Dado que \'e bastante comum queremos calcular ``os primeiros 10
termos da sequ\^encia'' ou ``os termos para $n$ valendo de $10$ a
$20$'', a linguagem incluir\'a um comando ``take'', cujo uso est\'a
exemplificado abaixo:
\begin{verbatim}
seq s;
s(n) <- 1 / ( n ^ 2 )

take(s, 1, 10); # Imprime os 10 primeiros termos se s
take(s, 10, 20); # Imprime os termos para n de 10 a 20
\end{verbatim}

Incluimos o comando take como um comando de repeti\c c\~ao pois em uma
linguagem como C, um while seria an\'alogo:
\begin{minted}[linenos=true, frame=lines]{cpp}
double s(double n) {
    return 1 / pow(n, 2);
}

int main() {
    // equivalente a take(s, 1, 10)
    int i = 1;
    while ( i <= 10 ) {
        printf("%lf ", s(i));
    }
    printf("\n");
    return 0;
}
\end{minted}
\subsection{Vincula\c c\~ao}
\subsubsection{Forma de vincula\c c\~ao}
A linguagem s\'o ir\'a permitir associa\c c\~ao expl\'icita, como
visto nos exemplos acima.

\subsubsection{Tempo de vincula\c c\~ao}
A vincula\c c\~ao ser\'a est\'atica, j\'a que n\~ao vemos necessidade
em alterar a mesma durante a execu\c c\~ao do programa.
\subsection{Sistema de tipos}
A linguagem ter\'a sobrecarga independente do contexto, j\'a que o
operador soma, por exemplo, pode ser tanto usado para somar valores
num\'ericos como para somar s\'eries.

Haver\'a coer\c c\~ao para tipos v\'alidos quando necess\'ario. Por
exemplo, a fun\c c\~ao $\sin$ recebe um valor real. Se um inteiro for
passado como par\^ametro, ele ser\'a promovido. Contudo, um erro
ser\'a disparado caso somemos um inteiro a uma s\'erie, por exemplo.

Por fim, o \'unico erro de tipos poss\'ivel ser\'a se declararmos uma
vari\'avel num\'erica e depois tentarmos atribuir a ela uma
sequ\^encia.


\end{document}