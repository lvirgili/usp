\documentclass[a4 paper, 12pt]{article}
\usepackage[brazilian]{babel}
\usepackage[utf8]{inputenc}
\usepackage[T1]{fontenc}
\usepackage{minted}
\usepackage{anysize}
\marginsize{3cm}{3cm}{2cm}{2cm}
\usemintedstyle{tango}
\newminted{cpp}{linenos, mathescape, frame=topline, numberblanklines=false}
\newmint{cpp}{frame=single}
\usepackage{algorithmic}
\usepackage[brazilian]{algorithm}

\renewcommand{\algorithmicrequire}{\textbf{Entrada:}}
\renewcommand{\algorithmicensure}{\textbf{Saída:}}
\renewcommand{\algorithmicend}{\textbf{fim}}
\renewcommand{\algorithmicif}{\textbf{se}}
\renewcommand{\algorithmicthen}{\textbf{então}}
\renewcommand{\algorithmicelse}{\textbf{else}}
\renewcommand{\algorithmicelsif}{\algorithmicelse\ \algorithmicif}
\renewcommand{\algorithmicendif}{\algorithmicend\ \algorithmicif}
\renewcommand{\algorithmicfor}{\textbf{para}}
\renewcommand{\algorithmicforall}{\textbf{para todos}}
\renewcommand{\algorithmicdo}{\textbf{faça}}
\renewcommand{\algorithmicendfor}{\algorithmicend}
\renewcommand{\algorithmicwhile}{\textbf{enquanto}}
\renewcommand{\algorithmicendwhile}{\algorithmicend}
\renewcommand{\algorithmicloop}{\textbf{loop}}
\renewcommand{\algorithmicendloop}{\algorithmicend\ \algorithmicloop}
\renewcommand{\algorithmicrepeat}{\textbf{repita}}
\renewcommand{\algorithmicuntil}{\textbf{enquanto}}
\renewcommand{\algorithmicprint}{\textbf{imprima}}
\renewcommand{\algorithmicreturn}{\textbf{retorne}}
\renewcommand{\algorithmictrue}{\textbf{verdadeiro}}
\renewcommand{\algorithmicfalse}{\textbf{falso}}


\renewcommand{\theFancyVerbLine}{
  \sffamily\textcolor[rgb]{0.5,0.5,0.5}{\scriptsize\arabic{FancyVerbLine}}}

\title{Projeto da Linguagem de Programa\c c\~ao}
\author{Lucas Virgili}
\date{}

\begin{document}
\maketitle
\tableofcontents
\section{Fase 3}
\subsection{Sintaxe}

Usaremos a nota\c c\~ao EBNF (\emph{Extended Backus-Naur Form}) para
definir a sintaxe da nossa linguagem de programa\c c\~ao. A sintaxe de
uma linguagem de programa\c c\~ao define as constru\c c\~oes que s\~ao
corretas dentro da linguagem.

\subsection{Palavras reservadas}

As seguintes palavras s\~ao reservadas dentro da nossa linguagem:

\begin{table}[h]
\begin{center}
\begin{tabular}{|c|c|}
\hline
Palavra             & Significado                                                         \\ \hline
seq                 & Usado para declarar vari\'aveis do tipo sequ\^encia \\ \hline
int                 & Usado para declarar vari\'aveis do tipo inteiro                     \\ \hline
real                & Usado para declarar vari\'aveis do tipo real                        \\ \hline
take                & Retorna elementos de uma sequ\^encia                \\ \hline
series              & Calcula a soma parcial de uma sequ\^encia           \\ \hline
sequence\_converges & Checa se uma sequ\^encia converge                   \\ \hline
series\_converges   & Checa se uma s\'erie converge                   \\ \hline
$\sin$              & Fun\c c\~ao seno                                                    \\ \hline
$\cos$              & Fun\c c\~ao cosseno                                                 \\ \hline
$\tan$              & Fun\c c\~ao tangente                                                \\ \hline
$\log$              & Fun\c c\~ao logaritmo com base $e$                                  \\ \hline
$\exp$              & Fun\c c\~ao exponencial                                             \\ \hline
sqrt                & Fun\c c\~ao ra\'iz quadrada                                         \\ \hline
\end{tabular}
\end{center}
\end{table}

\subsection{Regras l\'exicas}

Para definirmos as palavras v\'alidas na nossa linguagem, precisamos
definir o conjunto de caracteres que pertencem ao alfabeto da mesma,
os quais s\~ao combinados para formar as palavras.

Na nossa linguagem, as regras s\~ao as seguintes:
\begin{itemize}
\item Caracteres:
\begin{verbatim}
<letra> = "a" | "A" | "b" | "B" | "c" | "C" | "d" | "D" | "e" | "E" |
"f" | "F" | "g" | "G" | "h" | "H" | "i" | "I" | "j" | "J" | "k" | "K"
| "l" | "L" | "m" | "M" | "n" | "N" | "o" | "O" | "p" | "P" | "q" |
"Q" | "r" | "R" | "s" | "S" | "t" | "T" | "u" | "U" | "v" | "V" | "w"
| "W" | "x" | "X" | "y" | "Y" | "z" | "Z";

<char> = <letra> | "_";
\end{verbatim}

\item N\'umeros:
Primeiro precisamos definir os d\'igitos:
\begin{verbatim}
<dig> = 0 | 1 | 2 | 3 | 4 | 5 | 6 | 7 | 8 | 9;
\end{verbatim}

Agora, os tipos num\'ericos:

\begin{verbatim}
<int> = [ "-" ], dig, { dig };
<real> = [ "-" ], <int>, [ "."{ dig } ];
<numero> = <int> | <real>;
\end{verbatim}

\item Vari\'aveis:
Os identificadores de vari\'aveis podem conter n\'umeros ou
caracteres, mas n\~ao poder\~ao come\c car com n\'umeros. Logo,
\begin{verbatim}
<id> = <char>{ <char> | <dig> };
\end{verbatim}

\item Operadores:
Os operadores aritm\'eticos s\~ao os \'unicos presentes:
\begin{verbatim}
<op> = + | - | * | /;
\end{verbatim}

\item Literais:
Os literais num\'ericos j\'a foram definidos, basta un\'i-los com os
identificadores:
\begin{verbatim}
<lit> = <int> | <real> | <id>;
\end{verbatim}
\end{itemize}

\subsection{Regras sint\'aticas}
As regras sint\'aticas da nossa linguagem est\~ao definidas e
comentadas em EBNF abaixo:
\begin{verbatim}
(* Definicao da linguagem*)

(* Declaracoes. Tipos possiveis sao int, real ou seq *)
<tipo> = <int> | <real> | <seq>;
<seq> = <id> '(' <char> ')' = <expressao>;
<declaracoes> = <tipo> <identificador>;

(* Funcoes builtin*)
<func> = ( sin | cos | tan | log | exp | sqrt ) '(' <expressao> ')';

(* Definicao de uma expressao *)
<expressao> = ( <expressao> + <termo> ) |
              ( <expressao> - <termo> ) |
              ( <termo> );

<termo> = ( <termo> * <fator> ) |
          ( <termo> / <fator> ) |
          ( <fator> );

<fator> = <expressao> | <func> | <numero>;

(* take *)
<take> = take '('<id> , <int> ')';

(* sequence e series *)
<sequence> = <id> '(' <integer> ')'
<series> = series '(' <id> , <int>  ')';

(* Operadores de Convergencia *)
<seq_conv> = sequence_converges '(' <id>, <real> ')';
<series_conv> = series_converges '(' <id>, <real> ')';
\end{verbatim}

\subsection{Exemplo de programa}
Como j\'a hav\'iamos exemplificado na fase 1, um programa v\'alido na
nossa linguagem ser\'a algo como:
\begin{verbatim}
seq s;
s(n) = 1 / n ^ 2;
sequence(s, 3) = 0.125;
series(s, 3) = 0.875 ## calcula s_3;
sequence_converges(s, 0.000001);

series_converges(s, 0.00001)
\end{verbatim}

\subsection{S\'imbolos terminais}
Os s\'imbolos terminais aceitos j\'a foram mencionados na descri\c
c\~ao acima: literais, caracteres, operadores, fun\c c\~oes e
identificadores.


\end{document}